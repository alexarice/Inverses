\documentclass[presentation]{beamer}

\usetheme{Warsaw}
\author{Alex Rice}
\date{19/03/2020}
\title{Coinductive Invertibility in Higher Categories}

\usepackage{tikz-cd}
\usepackage{amsmath}
\usepackage{amsfonts}

\DeclareMathOperator{\id}{id}

\newcommand{\linv}[1]{{}^\star\!#1}
\newcommand{\rinv}[1]{#1^\star}
\newcommand{\inv}[1]{#1^{-1}}
\newcommand{\comp}{\star}


\begin{document}

\maketitle
\begin{frame}{Outline}
  \tableofcontents
\end{frame}

\section{Higher Categories}

\begin{frame}
  \frametitle{What is higher category?}
  In a regular category there are:
  \begin{itemize}
  \item A collection of objects;
  \item Between each pair objects there is a collection of morphisms between them.
  \end{itemize}
  \pause{}
  In higher category theory, we study cases where there is more structure on the collections of morphisms.
\end{frame}

\begin{frame}
  \frametitle{\(\omega\)-categories}
  In a 2-category, the collections of morphisms form categories.
  \pause{}

  In a 3-category, the collections of morphisms form 2-categories.
  \pause{}

  In an \(\omega\)-category, this continues.
\end{frame}

\begin{frame}
  \frametitle{Globular Sets}
  \(\omega\)-categories can take the shape of globular sets.
  \pause{}

  These are usually defined as presheafs.
  \pause{}
  \begin{definition}
    A \emph{globular set} \(\mathcal{G}\) is a collection of objects \(|\mathcal{G}|\) and for each \(x,y \in \mathcal{G}\) a globular set \(\mathcal{G}_{x,y}\).
  \end{definition}

  \pause{}
  An object \(f \in |\mathcal{G}_{x,y}|\) will be called a morphism between \(x\) and \(y\), and will be written \(f : x \to y\).

  \pause{}
  Let the objects of the globular set be it's \(0\)-cells, morphisms between these be \(1\)-cells, \(\dots\)

\end{frame}

\begin{frame}[fragile]
  \frametitle{Composition in infinity categories}
  \begin{block}{Composition of 1 cells}
    \begin{tikzcd}
      x \arrow[r, "f"] & y \arrow[r, "g"] & z
    \end{tikzcd}
  \end{block}
  \begin{block}{Composition of 2 cells}
    Codimension 1:
    \begin{tikzcd}
      \bullet \arrow[r, bend right=49, ""{name=S1}] \arrow[r, bend left=49, ""{name=T1, below}] & \bullet \arrow[r, bend right=49, ""{name=S2}] \arrow[r, bend left=49, ""{name=T2, below}] & \bullet
      \arrow[Rightarrow, from=S1, to=T1]
      \arrow[Rightarrow, from=S2, to=T2]
    \end{tikzcd}
    \pause{}

    Codimension 2:
    \begin{tikzcd}
      \bullet \arrow[r, bend right=90, ""{name=S1}] \arrow[r, ""{name=T1,below}, ""{name=S2}] \arrow[r, bend left=90, ""{name=T2, below}] & \bullet
      \arrow[Rightarrow, from=S1, to=T1]
      \arrow[Rightarrow, from=S2, to=T2]
    \end{tikzcd}
  \end{block}
\end{frame}

\begin{frame}
  \frametitle{Coherence for infinity categories}
  Infinity categories also have identity cells.

  For each \(n\)-cell \(f\) there is a cell \(\id_f : f \to f\).

  \pause{}
  Regular categories have associativity and unit laws. These are also present in \(\omega\)-categories.
\end{frame}

\begin{frame}
  \frametitle{Topological spaces}
  Topological spaces are nice examples of \(\omega\)-categories. Take a topological space \(X\).

  \pause{}
  \begin{itemize}
  \item 0 cells: \(X\)
  \item 1 cells: Paths between points
  \item 2 cells: Homotopies between paths
  \item \(\dots\)
  \end{itemize}

  \pause{}
  Identities are given by constant paths/homotopies.

  \pause{}
  Composition is given by path composition.
\end{frame}

\begin{frame}
  \frametitle{Fundamental \(\omega\)-groupoid}
  Similar to the topological space example, any type \(X\) forms an \(\omega\)-category.

  \pause{}
  \begin{itemize}
  \item 0 cells: terms of type \(X\);
  \item Higher cells: terms of equality types.
  \end{itemize}

  \pause{}
  Identities given by reflexivity proofs.

  \pause{}
  Composition is transitivity of equality.
\end{frame}

\begin{frame}
  \frametitle{\(\omega\)-\textbf{Cat}}
  \textbf{Cat}, the category of (small) categories, forms a 2-category with:
  \begin{itemize}
  \item 0-cells: Categories;
  \item 1-cells: Functors;
  \item 2-cells: Natural transformations.
  \end{itemize}

  \pause{}
  Similarly \textbf{2-Cat}, the category of 2-categories, forms a 3-categories.

  \pause{}
  The category of \(\omega\)-categories, \textbf{\(\omega\)-Cat}, is itself an \(\omega\)-category.
\end{frame}

\section{Equality in Higher Categories}

\begin{frame}
  \frametitle{Isomorphism}
\end{frame}

\begin{frame}
  \frametitle{Equivalence}
\end{frame}

\begin{frame}
  \frametitle{Inverses in higher categories}
\end{frame}

\begin{frame}
  \frametitle{Quasi-invertibility}
  \begin{definition}
    Given an \(n\)-cell \(f : x \to y\), a \emph{quasi-invertible} structure on \(f\) is a tuple \((\inv f, f_R, f_L, f_R{}I, f_L{}I)\) where:
    \begin{itemize}
    \item \(\inv f\) is an \(n\)-cell \(y \to x\);
    \item \(f_R\) is an \((n+1)\)-cell \(f \comp_0 \inv f \to \id_y\);
    \item \(f_L\) is an \((n+1)\)-cell \(\inv f \comp_0 f \to \id_x\).
    \item \(f_R{}I\) is a quasi-invertible structure on \(f_R\).
    \item \(f_L{}I\) is a quasi-invertible structure on \(f_L\).
    \end{itemize}
  \end{definition}
\end{frame}

\section{Forms of invertibility and results}

\begin{frame}
  \frametitle{Properties of higher categories}
\end{frame}

\begin{frame}
  \frametitle{String diagrams}
\end{frame}

\begin{frame}
  \frametitle{Properties of quasi-invertible structures}
  \begin{itemize}
  \item Given a quasi-invertible structure of \(f\), there exists a quasi-invertible structure on \(\inv f\).
  \item There is a quasi-invertible structure on any identity morphism.
  \end{itemize}
\end{frame}

\begin{frame}
  \frametitle{Limitations of quasi-invertibility}
\end{frame}

\begin{frame}
  \frametitle{Invertibility in type theory}
  Inverses of \(f : A \to B\):
  \begin{itemize}
  \item Quasi-invertible:
    \[ \mathsf{qinv}(f) : \Sigma_{g : B \to A} f \circ g \sim \id_B \times g \circ f \sim \id_A\]
  \item Bi-invertible:
    \begin{align*}
      &\mathsf{binv}(f) : \mathsf{linv}(f) \times \mathsf{rinv}(f) \\
      &\mathsf{linv}(f) : \Sigma_{g : B \to A} g \circ f \sim \id_A \\
      &\mathsf{rinv}(f) : \Sigma_{g : B \to A} f \circ f \sim \id_B
    \end{align*}
  \item Half-adjoint invertible:
    \[ \mathsf{ishai}(f) : \Sigma_{g : B \to A} \Sigma_{\eta : g \circ f \sim \id_A} \Sigma_{\epsilon : f \circ g \sim \id_B} \Pi_{x : A} f(\eta x) = \epsilon(f x) \]
  \end{itemize}
\end{frame}

\begin{frame}
  \frametitle{Bi-invertibility}
    Given an \(n\)-cell \(f : x \to y\), a \emph{bi-invertible} structure on \(f\) is a tuple \((\rinv f, \linv f, f_R, f_L, f_R{}BI, f_L{}BI)\) where:
  \begin{itemize}
  \item \(\rinv f\) is an \(n\)-cell \(y \to x\);
  \item \(\linv f\) is an \(n\)-cell \(y \to x\);
  \item \(f_R\) is an \((n+1)\)-cell \(f \comp_0 \rinv f \to \id_y\);
  \item \(f_L\) is an \((n+1)\)-cell \(\linv f \comp_0 f \to \id_x\).
  \item \(f_R{}BI\) is a bi-invertible structure on \(f_R\).
  \item \(f_L{}BI\) is a bi-invertible structure on \(f_L\).
  \end{itemize}
\end{frame}

\begin{frame}
  \frametitle{Properties of bi-invertible structures}
  \begin{itemize}
  \item Any quasi-invertible structure can be converted to a bi-invertible structure.
  \item Given a bi-invertible structures on a pair of compatible morphisms, there is a bi-invertible structure on their composite.
  \item Given a bi-invertible structure on \(f\), \(f, \rinv f, \linv f, \dots\) there are bi-invertible structures on both \(\rinv f\) and \(\linv f\).
  \end{itemize}
  \pause{}
  These are proved using coinduction and the results have been formalised in Agda using sized types.
\end{frame}

\begin{frame}
  \frametitle{Half-adjoint invertibility}
\end{frame}

\begin{frame}
  \frametitle{Adjoint equivalence}
\end{frame}

\begin{frame}
  \frametitle{Results}

\end{frame}

\begin{frame}
  \frametitle{Further work}
\end{frame}

\begin{frame}
  \frametitle{Bi-invertibility implies half-adjoint invertibility}
\end{frame}

\end{document}