\documentclass[presentation]{beamer}

\usetheme{Warsaw}
\author{Alex Rice}
\date{19/03/2020}
\title{Coinductive Invertibility in Higher Categories}

\DeclareMathOperator{\id}{id}

\newcommand{\linv}[1]{{}^\star\!#1}
\newcommand{\rinv}[1]{#1^\star}
\newcommand{\inv}[1]{#1^{-1}}
\newcommand{\comp}{\star}

\begin{document}

\maketitle
\begin{frame}{Outline}
  \tableofcontents
\end{frame}

\section{Higher Categories}

\begin{frame}
  \frametitle{What is higher category?}

\end{frame}

\begin{frame}
  \frametitle{Composition in higher categories}
\end{frame}

\begin{frame}
  \frametitle{Topological Spaces}
\end{frame}

\begin{frame}
  \frametitle{Fundamental \(\omega\)-groupoid}
\end{frame}

\begin{frame}
  \frametitle{\(\omega\)-\(\mathbf{Cat}\)}
\end{frame}

\section{Equality in Higher Categories}

\begin{frame}
  \frametitle{Isomorphism}
\end{frame}

\begin{frame}
  \frametitle{Equivalence}
\end{frame}

\begin{frame}
  \frametitle{Inverses in higher categories}
\end{frame}

\begin{frame}
  \frametitle{Quasi-invertibility}
  \begin{definition}
    Given an \(n\)-cell \(f : x \to y\), a \emph{quasi-invertible} structure on \(f\) is a tuple \((\inv f, f_R, f_L, f_R{}I, f_L{}I)\) where:
    \begin{itemize}
    \item \(\inv f\) is an \(n\)-cell \(y \to x\);
    \item \(f_R\) is an \((n+1)\)-cell \(f \comp_0 \inv f \to \id_y\);
    \item \(f_L\) is an \((n+1)\)-cell \(\inv f \comp_0 f \to \id_x\).
    \item \(f_R{}I\) is a quasi-invertible structure on \(f_R\).
    \item \(f_L{}I\) is a quasi-invertible structure on \(f_L\).
    \end{itemize}
  \end{definition}
\end{frame}

\section{Forms of invertibility and results}

\begin{frame}
  \frametitle{Properties of higher categories}
\end{frame}

\begin{frame}
  \frametitle{String diagrams}
\end{frame}

\begin{frame}
  \frametitle{Properties of quasi-invertible structures}
  \begin{itemize}
  \item Given a quasi-invertible structure of \(f\), there exists a quasi-invertible structure on \(\inv f\).
  \item There is a quasi-invertible structure on any identity morphism.
  \end{itemize}
\end{frame}

\begin{frame}
  \frametitle{Limitations of quasi-invertibility}
\end{frame}

\begin{frame}
  \frametitle{Invertibility in type theory}
  Inverses of \(f : A \to B\):
  \begin{itemize}
  \item Quasi-invertible:
    \[ \mathsf{qinv}(f) : \Sigma_{g : B \to A} f \circ g \sim \id_B \times g \circ f \sim \id_A\]
  \item Bi-invertible:
    \begin{align*}
      &\mathsf{binv}(f) : \mathsf{linv}(f) \times \mathsf{rinv}(f) \\
      &\mathsf{linv}(f) : \Sigma_{g : B \to A} g \circ f \sim \id_A \\
      &\mathsf{rinv}(f) : \Sigma_{g : B \to A} f \circ f \sim \id_B
    \end{align*}
  \item Half-adjoint invertible:
    \[ \mathsf{ishai}(f) : \Sigma_{g : B \to A} \Sigma_{\eta : g \circ f \sim \id_A} \Sigma_{\epsilon : f \circ g \sim \id_B} \Pi_{x : A} f(\eta x) = \epsilon(f x) \]


  \end{itemize}
\end{frame}

\begin{frame}
  \frametitle{Bi-invertibility}
\end{frame}

\begin{frame}
  \frametitle{Properties of bi-invertible structures}
\end{frame}

\begin{frame}
  \frametitle{Half-adjoint invertibility}
\end{frame}

\begin{frame}
  \frametitle{Adjoint equivalence}
\end{frame}

\begin{frame}
  \frametitle{Results}

\end{frame}

\begin{frame}
  \frametitle{Further work}
\end{frame}

\begin{frame}
  \frametitle{Bi-invertibility implies half-adjoint invertibility}
\end{frame}

\end{document}