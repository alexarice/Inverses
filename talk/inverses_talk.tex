\documentclass[presentation]{beamer}

\usetheme{Warsaw}
\author{Alex Rice}
\date{19/03/2020}
\title{Coinductive Invertibility in Higher Categories}

\begin{document}

\maketitle
\begin{frame}{Outline}
\tableofcontents
\end{frame}

\section{Higher Categories}

\begin{frame}
  \frametitle{What is higher category?}

\end{frame}

\begin{frame}
  \frametitle{Composition in higher categories}
\end{frame}

\begin{frame}
  \frametitle{Topological Spaces}
\end{frame}

\begin{frame}
  \frametitle{Fundamental \(\omega\)-groupoid}
\end{frame}

\begin{frame}
  \frametitle{\(\omega\)-\(\mathbf{Cat}\)}
\end{frame}

\section{Equality in Higher Categories}

\begin{frame}
  \frametitle{Isomorphism}
\end{frame}

\begin{frame}
  \frametitle{Equivalence}
\end{frame}

\begin{frame}
  \frametitle{Inverses in higher categories}
\end{frame}

\begin{frame}
  \frametitle{Quasi-invertibility}
\end{frame}

\section{Forms of invertibility and results}

\begin{frame}
  \frametitle{Properties of higher categories}
\end{frame}

\begin{frame}
  \frametitle{String diagrams}
\end{frame}

\begin{frame}
  \frametitle{Properties of quasi-invertible structures}
\end{frame}

\begin{frame}
  \frametitle{Limitations of quasi-invertibility}
\end{frame}

\begin{frame}
  \frametitle{Invertibility in type theory}
\end{frame}

\begin{frame}
  \frametitle{Bi-invertibility}
\end{frame}

\begin{frame}
  \frametitle{Properties of bi-invertible structures}
\end{frame}

\begin{frame}
  \frametitle{Half-adjoint invertibility}
\end{frame}

\begin{frame}
  \frametitle{Adjoint equivalence}
\end{frame}

\begin{frame}
  \frametitle{Results}
\end{frame}

\begin{frame}
  \frametitle{Further work}
\end{frame}

\begin{frame}
  \frametitle{Bi-invertibility implies half-adjoint invertibility}
\end{frame}

\end{document}